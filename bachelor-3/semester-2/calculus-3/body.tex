\section{Sequences}

\subsection{Sequence}
\textbf{Sequence} ordered collection of numbers defined by function f. Usually denoted $a_{n}$.

$a_{n}=f_{n}$ known as \textbf{terms}

$n$ is the \textbf{sequence}

\subsection{Converge and Divergence}
if $\lim_{x\to\infty}f(x)$ exists, then $a_{n}=f(n)$ converges to the same limit

\begin{equation}
	\label{}
\lim_{n\to\infty}a_{n}=\lim_{x\to\infty}f(x)	
\end{equation}

generally testing this is straightforward.

\subsection{Function raised to a function- incomplete}

\begin{equation}
	\label{}
\lim_{x\to\infty}	(1+\frac{a}{n})^{n}	
\end{equation}
take natural log, move exponent to coefficient, and find limit
\begin{equation}
	\label{}
n\ln(1+\frac{a}{n}	)
\end{equation}
can be formatted as a fraction for L'Hopital's rule
\begin{equation}
	\label{}
\frac{\ln(1+\frac{a}{n})}{\frac{1}{n}}	
\end{equation}



\begin{equation}
	\label{}
e^{n}	
\end{equation}


\subsection{Geometric sequence and series }
for $a_{n}=cr^{n}$
\begin{enumerate}
	\item if $|r|<1$,
	\item if $r>1$ then $\lim_{n\to\infty}a_{n}=\infty$
	\item if $r=0$, $\lim_{n\to\infty}=0$
\end{enumerate}

\hfill
\hfill


\textbf{Geometric Series}



\begin{enumerate}
	\item if $|r|>1$, diverges
	\item if $|r|<1$, converges to $a\frac{1}{1-r}$
\end{enumerate}

\subsection{trigonometric functions, divergence by oscillation}
\begin{equation}
	\label{}
	\sin(x),\cos(x)
\end{equation}
diverge by \textbf{oscillation}

\subsection{Squeeze Theorem}
\begin{equation}
	\label{}
a_{n}=e^{-2n}\cos(n)
\end{equation}
squeezing $\cos(n)$;
\begin{equation}
	\label{}
-1\leq \cos(n) \leq 1	
\end{equation}
\begin{equation}
	\label{}
-e^{-2n}\leq -e^{2n}\cos(n) \leq -3^{-2n}	
\end{equation}
because $-e^{2n}$ approaches 0 when  $\lim_{n\to\infty}$, the whole function converges

\subsection{Bounded, Monotonic}
\textbf{Bounded}, has a maximum or minimum value

\textbf{Monotonic}, either increasing or decreasing.

\begin{enumerate}	
	\item If increasing and bounded above, converges.

\item If decreasing and bounded below, converges.

\item If converges, bounded
\end{enumerate}

To determine if a series is monotone, take derivative. This sometimes can indicate monotonicity.


\section{Series}
\subsection{series}

\textbf{Series}, adding every term in a sequence.


\[ \sum_{n=1}^{\infty} a_{n} \]

$S_{n}$ corresponds to $a_{n}$, but rather than that place in the sequence, it is the sum of every previous term and that term in the series.


Some have infinite sums were the answer approaches a value.

\begin{equation}
	\label{}
	S_{n}= \sum_{n=1}^{\infty} 0.1^{n} \to 0.11111\ldots \to \frac{1}{9}
\end{equation}

Series can sometimes be written as sequences.

\begin{equation}
	\label{}
S_{n}=\sum_{n=1}^{\infty} \frac{1}{4k^{2}-1}=\frac{n}{2n+1}	
\end{equation}

If the sequence of sums, $S_{n}$ diverges, then the series diverges.


Intuition can be deceiving
\begin{equation}
	\label{}
\sum_{n=1}^{\infty}\frac{1}{n}
\end{equation}
Diverges

\subsection{Sum does not start at n=0}
\begin{equation}
	\label{}
\sum_{k=2}^{\infty}a r^k	
\end{equation}

equivalent to

\begin{equation}
	\label{}	
	\sum_{k=0}^{\infty}a r^{k+2}
\end{equation}

\subsection{Telescoping Series}
A \textbf{telescoping series} is one in which the terms cancel. One is often left with an initial term and a final term that has not yet canceled. This latter will have n in it.\ not so if n goes to infinity.

Some things don't look like they're telescoping but require partial fractions

\subsection{Diverge and Integral Test}
If $\sum_{n=1}^{\infty}a_{n}$ converges, then $ \lim_{n\to\infty} a_{k}=0$. Equivalently if $ \lim_{n\to\infty} a_{n}\neq 0 $, series diverges.

If $\lim_{n\to\infty}\to0$, inconclusive. If $ \lim_{n\to\infty}=a\in\mathbb{R}$, converges



\hfill


\textbf{Integral Test} let $a_{n}=f_{n}$, where $f$ is positive, decreasing, and continuous. If $\int f(n)$ converges, $\sum_{n=1}^{\infty}a_{n}$ converges.

Ex.
\begin{equation}
	\label{}
\sum_{n=1}^{\infty}\frac{1}{n}
\end{equation}

make sure positive, decreasing, continuous 

\begin{equation}
	\label{}
	\int_{1}^{\infty} \frac{1}{n}
\end{equation}
\begin{equation}
	\label{}
	\ln(n)|^{\infty}_{1}
\end{equation}
\begin{equation}
	\label{}
	\ln(\infty) - \ln(1) = \infty - 0
\end{equation}
diverges.

\subsection{Convergence of the p-series}

The p-series
\begin{equation}
	\label{}
\sum_{n=1}^{\infty}\frac{1}{k^{p}}	
\end{equation}
converges when $p>1$ and diverges when $p\leq1$


\subsection{Comparison Tests}
Assume there exists $M>0$ such that $0\leq a_{n}\leq b_{n}$ for $n\geq M$. Past a certain value, $M$, b is forever greater than a.


\begin{enumerate}
	
	\item if $\sum_{n=1}^{\infty}b_{n}$ converges, then $\sum_{n=1}^{\infty} a_{n}$ also converges.

\item If $\sum_{n=1}^{\infty}a_{n}$ diverges, $\sum_{n=1}^{\infty}b_{n}$ also diverges
\end{enumerate}

\subsection{Limit Comparison}

Let $a_{n},b_{n}$ be positive sequences. Assume the following limit exists:
\begin{equation}
	\label{}
	L= \lim_{n\to\infty} \frac{a_{n}}{b_{n}}
\end{equation}


\begin{enumerate}
	\item if $L>0$ then $\sum_{n=1}^{\infty} a_{n}$ converges iff $\sum_{n=1}^{\infty}b_{n}$ converges.	

	\item if $L=\infty$ and $\sum a_{n}$ converges, then $\sum b_{n}$ converges.
	

	\item if $L=0$ and $\sum b_{n}$ converges, then $\sum a_{n}$ converges.

\end{enumerate}


\subsection{Alternating Series and Absolute Convergence}

\textbf{Alternating Series}, a series whose terms switch between positive and negative
\begin{equation}
	\label{}
	\sum_{n=1}^{\infty}(-1)^{n}b_{n} \mbox{, or } \sum_{n=1}^{\infty}(-1)^{n+1}b_{n}	
\end{equation}
for $b_{n}\geq 0$

\textbf{Absolute Convergence} The series $\sum a_{n}$ \textbf{converges absolutely} if $\sum |a_{n}|$ converges.

If $a_{n}$ has absolute convergence, $a_{n}$ converges

\hfill
\hfill

\textbf{alternating series test}
if $b_{n}$ is a positive sequence that is decreasing and converges to 0, then
\begin{equation}
	\label{}
S=\sum_{n=1}^{\infty}(-1)^{n-1}b_{n}
\end{equation}
converges.

\hfill


Furthermore,
\begin{equation}
	\label{}
	0<S<b_{1} \mbox{, and } S_{2N}<S<S_{2N}\mbox{, } N\geq 1	
\end{equation}


\subsection{Conditional Convergence}
An infinite series $\sum a_{n}$ converges conditionally if $\sum a_{n}$ converges but $\sum |a_{n}|$ diverges.


\subsection{Ratio and Root Tests}
\textbf{Ratio Test} * good for factorials
\begin{equation}
	\label{}
	\rho=\lim_{n\to\infty}|\frac{a_{n+1}}{a_{n}}|	
\end{equation}

\begin{enumerate}
	\item if $\rho<1, \sum a_{n}$ converges absolutely
	\item if $\rho>1,\sum a_{n}$ diverges
	\item if $\rho=1$, the test is inconclusive
\end{enumerate}
\textbf{Root Test}
\begin{equation}
	\label{}
	L=\lim_{n\to\infty}\sqrt[k]{|a_{n}|}
\end{equation}


\begin{enumerate}
	\item if $L<1, \sum a_{n}$ converges absolutely
	\item if $L>1,\sum a_{n}$ diverges
	\item if $L=1$, the test is inconclusive
\end{enumerate}

\subsection{Convergence and Function Approximation. Power Series}
Power series can sometimes be approximated by the less intensive solution to geometric series with $|r|<1$
\begin{equation}
	\label{}
\frac{1}{1-x}=\sum_{n=1}^{\infty} x^{k}
\end{equation}

For
\begin{equation}
	\label{}
	\sum_{n=0}^{\infty}{(ax-b)}^{n}	
\end{equation}
This approximation converges only when $|ax-b|<1$. Solving this and finding mid point can find the radius of convergence

\hfill
\hfill
\hfill

Not all power series have the form of a geometric series. Coefficients $c_{k}$ might not be the same for all terms. The \textbf{Ratio Test} is often useful in these situations.

\begin{equation}
	\label{}
	a_{k}=\sum_{k=1}^{\infty} \frac{(x-7)^{k}}{k}	
\end{equation}
evaluate:
\begin{equation}
	\label{}
L=\lim_{n\to\infty}|\frac{a_{k+1}}{a_{k}}|
\end{equation}
\begin{enumerate}
	\item if $L<1$, convergence
	\item if $L>1$, divergence
	\item if $L=1$, inconclusive
\end{enumerate}

The values that make $L=1$ must be evaluated to find specific character


\subsection{Taylor Series}

\begin{equation}
	\label{}
P_{n}(x)=\sum_{k=0}^{n}\frac{f^{(k)}(a)}{k!}(x-a)^{k}
\end{equation}
A Maclaurin series is a Taylor series centered at $x=0$



\section{Polar, Cylindrical and Spherical Coordinates}

In a rectangular coordinate system:
a point at (x,y) has length r and is above the horizontal axis at $\theta$


x and y can be represented
\begin{equation}
	\label{}
x=r\cos\theta, y=r\sin\theta	
\end{equation}
and
\begin{equation}
	\label{}
r^{2}=x^{2}+y^{2}	
\end{equation}
\begin{equation}
	\label{}
\tan\theta=\frac{y}{x}	
\end{equation}

\subsection{General Polar Equation Forms}
\textbf{cardioid}
\begin{equation}
	\label{}	
r=a(1\pm \cos\theta)
\end{equation}
\begin{equation}
	\label{}
r=a(1\pm\sin\theta)	
\end{equation}
\textbf{rose}
\begin{equation}
	\label{}
r=a\cos(b\theta), r=a\sin(b\theta)	
\end{equation}


\subsection{Cylindrical and Spherical Coordinate Systems}

\textbf{Cylindrical Coordinate Systems}, $P(x,y,z)$ is represented by $P(r,\theta,z)$.
if
\begin{equation}
	\label{}
z=r	
\end{equation}
A cone has been formed.

\textbf{Spherical Coordinates}
$P(\rho,\theta,\phi)$
\begin{enumerate}
	\item $\rho$ is the distance between P and origin.
	\item $\theta$ is the angle used in cylindrical or polar coordinates
	\item $\phi$ is the angle between the z axis and the line segment OP, where  is the origin and $0\leq\phi\leq\pi$
\end{enumerate}

\subsection{Between spherical, cylindrical, and rectangular coordinates}

\begin{equation}
	\label{}
x=\rho\sin\phi\cos\theta	
\end{equation}


\begin{equation}
	\label{}
y=\rho\sin\phi\sin\theta	
\end{equation}

\begin{equation}
	\label{}
z=\rho\cos\phi	
\end{equation}

\begin{equation}
	\label{}
	\sqrt{x^{2}	+y^{2}}=\rho\sin\phi
\end{equation}


and

\begin{equation}
	\label{}
	\rho^{2}=x^{2}+y^{2}+z^{2}
\end{equation}
\begin{equation}
	\label{}
\tan\theta=\frac{y}{x}	
\end{equation}
\begin{equation}
	\label{}
	\phi=\arccos\big(\frac{z}{\sqrt{x^{2}+y^{2}+z^{2}}}	\big)
\end{equation}

Relationship between cylindrical and spherical coordinates

\begin{equation}
	\label{}
r=\rho\sin\phi
\end{equation}
\begin{equation}
	\label{}
\theta=\theta	
\end{equation}
\begin{equation}
	\label{}
z=\rho\cos\phi	
\end{equation}
and
\begin{equation}
	\label{}
	\rho=\sqrt{r^{2}+z^{2}}
\end{equation}
\begin{equation}
	\label{}
	\phi=\arccos\big(\frac{z}{\sqrt{r^{2}+z^{2}}}\big)	
\end{equation}

\section{Vectors}
\begin{equation}
	\label{}
r(t)=f(t)i+g(t)j= \langle f(t),g(t)\rangle
\end{equation}
This makes a function. Following the vector across $t$ makes its own curve.

\textbf{Initial point:} $(x_{0},y_{0})$.


\textbf{Terminal point:} $(x_{1},y_{1})$

A vector is in \textbf{standard position} if the initial point is at the origin When graphing we usually graph vectors in the domain of the function in standard position


A \textbf{plane curve} is created by a function of $\hat{i},\hat{j}$

A \textbf{space curve} is created by a function of $\hat{i},\hat{j},\hat{k}$


\subsection{Parametrization Parametric}
Line segment
\begin{equation}
	\label{}
	<x_{1}+t(x_{2}-x_{1}),y_{1}+t(y_{2}-y_{1})>	
\end{equation}
Circle
\begin{equation}
	\label{}
<r\cos t,r\sin t>	
\end{equation}


\subsection{Differentiating and operating on Vector Valued Functions}

Differentiating a vector value function at a point gives a tangent vector at that point
\begin{equation}
	\label{}
r(t)=f(t)i+g(t)j	
\end{equation}
\begin{equation}
	\label{}
r'(t)=f'(t)i+g'(t)j	
\end{equation}

\textbf{dot product}: given $(x_{1},y_{1})$ and $(x_{2},y_{2})$
\begin{equation}
	\label{}
u\dot v = x_{1}x_{2}+y_{1}y_{2}
\end{equation}	

\textbf{cross product:} say $u_{1}=(a_{1},a_{2},a_{3})$, and $v_{1}=(b_{1},b_{2},b_{3})$. Magnitude of cross product is area bounded by vectors.
\begin{equation}
	\label{}
\begin{bmatrix}
	i&j&k\\
	a1&a2&a3\\
	b1&b2&b3
\end{bmatrix}	
\end{equation}
\begin{equation}
	\label{}
	u \times  v=(a2b3-b2a3)i - (a1b3-b1a3)j + (a1b2-b1a2)k
\end{equation}



\subsection{Length of a Vector Valued Function}
fr $r(t)=f(t)i+g(t)j\cdots +z(t)z$
\begin{equation}
	\label{}
	\int\sqrt{f'(t)^{2}+g'(t)^{2}\cdots +z'(t)^{2}}dt=\int ||r'(t)||dt
\end{equation}

\textbf{Arclength Function}


\begin{equation}
	\label{}	
s(t)=\int ||r'(u)||du
\end{equation}

\begin{equation}
	\label{}	
\frac{ds}{dt}
\end{equation}


\subsection{Unit Tangent Vector}
\begin{equation}
	\label{}
T(t)=\frac{r'(t)}{||r'(t)||}	
\end{equation}



\subsection{Curvature}
often measured in relation to \textbf{radius of curvature.} If a circle where overlaid at that point, what would the radius be to match the curve.
\begin{equation}
	\label{}
	\kappa=\frac{||T'(t)||}{||r'(t)||}=\frac{|y''|}{[1+(y')^{2}]^{\frac{3}{2}}}=||\frac{dT}{ds}||=||T'(s)||
\end{equation}
\begin{equation}
	\label{}
	\kappa=\frac{||r'(x)\times r''(x)||}{||r'(x)||}
\end{equation}

\subsection{Principal Unit Normal Vector and Binormal Vector}
\textbf{Principal Unit Normal Vector:} vector of length one perpendicular to curve at a point.
\begin{equation}
	\label{}
N(t)=\frac{T'(t)}{||T'(t)||}	
\end{equation}
\textbf{Binormal Vector} is orthogonal to $T$ and $N$


\begin{equation}
	\label{}
T(t)\times N(t)	
\end{equation}
\begin{equation}
	\label{}
||B||=||T \times N||=||T||||N||\sin\theta=1	
\end{equation}


\subsection{Acceleration}
\begin{equation}
	\label{}
	a(t)=v'(t)\cdot T(t) + [v(t)]^{2}\cdot\kappa\cdot N(t)	
\end{equation}

\section{Integrals}
for $f$ continuous on a rectangular region $a\leq x\leq b$ and $c\leq y\leq d$. Either order of standard double integration will work
\begin{equation}
	\label{}
\iint f(x,y)dA=\lim_{m,n\to\infty}\sum^{m}_{i=1}\sum^{n}_{j=1}f(x,y)
\end{equation}
\begin{equation}
	\label{}
	\mbox{Volume}=\int A(x)dx	
\end{equation}
with $A(x)$ being a function for area, $A(x)=\int f(x)dy$


\subsection{Non-rectangular}
Some shapes are bounded by two functions. If they are functions of x, initially integrate with respect to y, with the functions as the bounds of the integral.


Let $R$ be a regin bounded below and above by the graphs of their continuous functions $y=g(x)$ and $y=h(x)$, and by the lines a=x and x=b. If $f$ is continuous of $R$, then
\begin{equation}
	\label{}
\iint f(x,y)dA=\iint f(x,y)dx dy	
\end{equation}

\subsection{2 functions}

\begin{equation}
	\label{}	
\iint g(x,y)-f(x,y)A
\end{equation}

use the intersection of these forms projected on xy axis as the bounds of integrals.

\subsection{Integration with Polar Coordinates}

$R={(r,\theta) : 0\leq a \leq r \leq b, \alpha \leq\theta\leq\beta}$
\begin{equation}
	\label{}	
\iint f(r,\theta)dA=\int_{\alpha}^{\beta}\int_{a}^{b} f(r,\theta) \cdot r dr d\theta
\end{equation}
where $f(r,\theta)$ is z

This can be used for overlapping circles and a lot of other things. The main difference being the bounds of the integrals. One may have to ad further integrated integrals

\subsection{Triple integrals}

\begin{equation}
	\label{}	
\iiint f(x,y,z)dV
\end{equation}

\subsection{Triple Integrals in Cylindrical and Spherical Coordinates}

\begin{equation}
	\label{}	
	\Delta V=\rho^{2} \sin\phi\cdot d\rho\cdot d\phi\cdot d\theta
\end{equation}
\begin{equation}
	\label{}
\iiint=f(\rho,\theta,\phi)\rho^{2}\sin\phi
\end{equation}

\section{Vector Fields}

insane. Can be used to model all sorts of fields. 



A vector field $F$ in $\mathbb{R}^{n}$ is an assignment of an n dimensional vector $F(x,y,etc)$ t each point of a subset $D$ in $\mathbb{R}$

A vector field is \textbf{continuous} if both components are continuous. 

\hfill

two kinds of vector fields. In a \textbf{radial field} all vectors either point toward or away from the origin

\hfill

\textbf{Rotational field} is tangent t a circle with radius $r=\sqrt{x^{2}+y^{2}}$. \textbf{Dot Product} is zero.

\hfill
\hfill

a \textbf{Unit vector field} is a field in which every vector has magnitude 1. 

\subsection{Normalizing a Vector field}

\begin{equation}
	\label{}
	F=<P,Q,R>
\end{equation}
unit field:
\begin{equation}
	\label{}
	\frac{F}{||F||}
\end{equation}

\subsection{Gradient}

\begin{equation}
	\label{}
	\mbox{grad}f=\nabla f=<f'x,f'y>	
\end{equation}

A field is a \textbf{gradient field} or \textbf{conservative vector field} if there is a single scalar function f such that $\nabla f=F$ f must be a function where if differentiated for each component $(x,y,etc,)$, it yields the components of F in accordance to variable differentiated for.

\hfill
\hfill

$f$ is called a \textbf{potential function}


\textbf{Cross Partial Property of Conservative Vector fields}

If $F(x,y)=\langle P(x,y),Q(x,y)\rangle$ is a conservative vector field then $\frac{\partial P}{\partial y}=\frac{\partial Q}{\partial x}$. 


\hfill

This can be used to show a field is conservative, not vice versa.


\subsection{Vector Line Integral}
the \textbf{Vector Line Integral} of a vector field $F$ along an oriented smooth curve $C$ is 
\begin{equation}
	\label{}
\int_{C}F \cdot T ds = F(r(t)) \cdot \frac{r'(t)}{||r'(t)||}\cdot ||r'(t)||dt
\end{equation}
\begin{equation}
	\label{}
	\int_{C}=F\cdot Tds = \int_{a}^{b}F(r(t))\cdot r'(t)dr
\end{equation}
\begin{equation}
	\label{}
\int_{C}F\cdot T ds = \int_{C}F\cdot dr	
\end{equation}

\textbf{Piecewise Smooth Function}
a function made of a finite number of smooth curves

\begin{equation}
	\label{}
\sum_{m=1}^{n}	\int_{C_{m}}F\cdot ds
\end{equation}


\subsection{Flux}
\begin{equation}
	\label{}
\int_{C}F\cdot N ds	
\end{equation}

\begin{equation}
	\label{}
\int_{C}F(r(t))\cdot n(t)dt	
\end{equation}
All these variables are vectors


\begin{equation}
	\label{}
n=<y',-x'>	
\end{equation}

\subsection{Circulation}
Circulation of F along C: line integral of F along oriented \textbf{closed} curve

\begin{equation}
	\label{}
\oint_{C}F\cdot T ds	
\end{equation}

\begin{equation}
	\label{}
\int F(r(t))\cdot r'(t)	
\end{equation}

\textbf{Simple Curves} do not cross themselves.

A region D is a \textbf{connected region} for any two points if there is a path where the trace is entirely within D. A region is \textbf{simply connected} if you can shrink it to a straight line. If there is a hole/excepted area within the region it is not simply connected.

\subsection{Fundamental Theorem for Line Integrals}
C must be a piecewise smooth curve

\begin{equation}
	\label{}
\int_{C}\nabla f \cdot dr = f(r(b))-f(r(a))	
\end{equation}


\textbf{Gradient fields} are \textbf{path independent}

\subsection{Green's Theorem}

Let D be an open, simply connected region with a boundary curve C that is piecewise smooth, simple closed curve oriented counterclockwise. Only for 2 dimensional vector fields.
\begin{equation}
	\label{}
	\oint_{C}F\cdot dr = \oint_{c}Pdx+Qdx = \iint_{D}(Q_{x}-P_{y})dA=\int_{C}=F\cdot T ds
\end{equation}
\begin{equation}
	\label{}
\iint_{D}(Q_{X}-P_{y})dA	
\end{equation}

Given an equation that satisfies, identify P, which is with $dx$, and Q, which is with $dy$, and then put them in the form. If it is going clockwise, make it negative.

\hfill

If $Q_{x}-P_{y}$=1, $dA$ is integrated and is equal to the initial integral.

\hfill
\hfill

\subsection{Parametrize an Ellipse}
\begin{equation}
	\label{}
<a\cos t,b\sin t>	
\end{equation}
with a as top radius in x, b top radius in y

\subsection{Flux Form of Green's Theorem}

Let D be an open, simply connected region with a boundary curve C that is piecewise smooth, simple closed curve oriented counterclockwise. Only for 2 dimensional vector fields.
\begin{equation}
	\label{}
\oint_{C}F\cdot N ds = \iint_{D}(P_{x}+Q_{y})dA	
\end{equation}


\subsection{Harmonic functions}

A \textbf{source free vector field} is a conservative field but with flux instead.


\hfill
\hfill

Conservative and source free vector fields on simply connected domain: any potential function satisfied Laplace's Equation: $f_{xx}+f_{yy}=0$. $f$ is a harmonic function.

\subsection{Non simply connected regions}
Split the integrals up until they are simply connected



\subsection{Divergence}
Divergence measures the `genesis' of a certain point, $div F(x,y)$. Divergence is negative if it `flows in', and positive if its generative


\subsection{Gradient Operator}
\begin{equation}
	\label{}

\nabla=<\frac{\partial}{\partial x},\frac{\partial}{\partial y},\frac{\partial}{\partial z}>
\end{equation}
\begin{equation}
	\label{}
	\mbox{div}F=\nabla \cdot F	
\end{equation}

Let $F=<P,Q>$ be a simply connected vector field.
\begin{equation}
	\label{}
	\mbox{div}F=0	
\end{equation}
\textbf{iff} $F$ is source-free

\subsection{Curl}
for $F=<P,Q,R>$, a vector field whose component derivatives all exist,
\begin{equation}
	\label{}
	\mbox{curl}F=\nabla \times  F	
\end{equation}
\begin{equation}
	\label{}
\begin{bmatrix}
	i&&j&&k\\
	\frac{\partial}{\partial x}&&\frac{\partial}{\partial y}&&\frac{\partial}{\partial z}\\
	P&&Q&&R
\end{bmatrix}	
\end{equation}

\begin{equation}
	\label{}
	\mbox{div (curl (F))}=0	
\end{equation}

for a conservative vector field curl$F=0$


divergence of a gradient is 0
\end{document} 

