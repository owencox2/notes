\section{Newton's Laws}



\begin{enumerate}
	\item If $\vec F=0$, and $\vec v$ is constant, then the reference frame is \textbf{inertial}
	
	\item In \textbf{inertial frames}
		$\vec{F}=m\frac{d\vec{}v}{dt}$, where $\vec{p}=mv$. $F=\frac{d\vec{p}}{dt}$
	\item $\vec{F_{21}}=-\vec{F_{12}}$
\end{enumerate}

It is often useful to use this form for when mass is changing.
\begin{equation}
	\label{}
	F=\frac{d\vec{p}}{dt}	
\end{equation}



For derivative notation:


\begin{equation}
	\label{}
	\vec F=m\ddot{\vec{r}}
\end{equation}
\begin{equation}
	\label{}
	F_{x}=m\ddot{x}=m\frac{d^{2}x}{dt^{2}}	
\end{equation}
same is true for y and z.


For these equations a number of initial conditions are required to solve the problems. Initial velocity and position.

\subsection{Inertial and Non-Inertial Reference frames}
in any large-scale earth-based problem, there is not an inertial reference frame due to the curvature and \textbf{rotation} of the earth and the force of gravity. These laws as written do not quite apply.

\subsection{Galilean Transformation}
from (x,y,z) to (x',y',z') moving with velocity v relative to (x,y,z)
\begin{enumerate}
	\item $x=x'-v_{x}t$
	\item $y=y'-v_{y}t$
	\item $z=z'-v_{z}t$
\end{enumerate}



acceleration is invariant under Galilean transformation

\hfill
\hfill

In modern physics, the approach is to first define \textbf{invariant properties}, which then determines what the physics is. From acceleration is invariant under Galilean transformation, Newton's laws can be determined.



\subsection{Polar Coordinates}

\begin{equation}
	\label{}
	\vec{v}\equiv \vec{\dot{r}}=\dot{r}\hat{r}+r\dot{\phi}\hat{\phi}	
\end{equation}
\begin{equation}
	\label{}	
	\vec{a}\equiv \vec{\ddot{r}}=(\ddot{r}-r\dot{\phi^{2}})\hat{r}+(r\ddot{\phi}+2\dot{r}\dot{\phi})\hat{\phi}
\end{equation}

For solving force of tension on string of ball with uniform circular motion. 


\subsection{Note: Technique for solving differential equations:}

For
\begin{equation}
	\label{}
\frac{df}{dt}=f	
\end{equation}
\begin{equation}
	\label{}
\int\frac{1}{f}df=\int dt	
\end{equation}
and so on, solving for f.

\subsection{Drag}
\begin{equation}
	\label{}
f(v)=bv+cv^{2}	
\end{equation}
\begin{equation}
	\label{}
	\vec{f}=-f(v)\hat{v}	
\end{equation}
\begin{equation}
	\label{}
V_{ter}=\frac{mg}{b}	,	
\tau=\frac{m}{b},
V_{ter}=\sqrt{\frac{mg}{c}}
\end{equation}

\section{Conservation}
\textbf{Conservation: } does not change with respect to time.

\begin{enumerate}
	\item Momentum is conserved under no net forces
	\item Energy is conserved if you keep track of it \textbf{all}
	\item Angular momentum is conserved under no net torque
	\item charge is conserved if you keep track of it all
\end{enumerate}

\subsection{Energy}
\begin{equation}
	\label{kenergy}
	T=\frac{1}{2}mv^{2}
\end{equation}
\begin{equation}
	\label{}
	\frac{dT}{dt}=\vec{F}\cdot\vec{v}=\vec{F}\cdot\frac{d\vec{r}}{dt}
\end{equation}
\begin{equation}
	\label{}
	dT=\vec{F}\cdot d\vec{r}
\end{equation}
\begin{equation}
	\label{}
	dT=\sum_{i}\vec{F}_{i}\cdot d\vec{r}	
\end{equation}
\begin{equation}
	\label{}
	dW_{i}=\vec{F_{i}}\cdot d\vec{r}
\end{equation}

If W done by force $\vec{F}$ is path dependent then nonconservative.
\begin{equation}
	\label{}
	W=\int_{C}\vec{F}\cdot d\vec{r}
\end{equation}
\begin{equation}
	\label{}
	\nabla \times \vec{F} =0
\end{equation}
in a conservative field
\begin{equation}
	\label{}
	\vec{F}=-\nabla U
\end{equation}

\subsection{Stoke's Theorem}
\begin{equation}
	\label{}
	\oint\vec{F}\cdot d\vec{r}=\int\nabla \times \vec{F}\cdot d\vec{A}
\end{equation}

If curl (f), force is conservative.

\hfill
\hfill

If force is conservative every point has a unique value that is path independent

U (x) is the \textbf{potential function} (see calc notes)

\subsection{Momentum chapter equations}
Rockets
\begin{equation}
	\label{}
	m\dot{v}=-\dot{m}v_{ex}+F^{ext}
\end{equation}
Center of Mass
\begin{equation}
	\label{}
	R=\frac{1}{M}\sum_{\alpha=1}^{N}m_{\alpha}r_{\alpha}
\end{equation}
Angular Momentum
\begin{equation}
	\label{}
	l=r \times p
\end{equation}
\begin{equation}
	\label{}
	\dot{\phi}=\omega=\frac{l}{mr^{2}},\quad l=\omega mr^{2}=\dot{\phi}mr^{2}
\end{equation}


\subsection{General Solution for a 1 dimensional problem}
\begin{equation}
	\label{}
	\int_{s_{0}}^{s}\frac{ds}{\sqrt{2[\frac{E}{m}-\frac{1}{m}U(s)]}}=\int_{t_{0}}^{t}dt
\end{equation}
Graphing the potential curve $U(S)$ gives information of how the system will react. Graphing total mechanical energy, the intersection are the \textbf{turning points.} \textbf{Equilibria points} are when $\frac{dU}{ds}=0$.
If $\frac{d^{2}}{ds^{2}}>0$, system is stable. If $\frac{d^{2}U}{ds^{2}}<0$, system is unstable, and will readily go to the lower state.

\hfill
\hfill

For a force $\vec{F}=-\nabla U(r,\theta,\phi)$ where the theta and phi components are zero, the system is \textbf{spherically symmetrical}, and it can be considered a \textbf{one-dimensional problem}. $\vec{F}$ reduces to $\vec{F}=-\frac{dU}{dr}\hat{r}$, known as a \textbf{central force}. 

\subsection{Force Law}
\begin{equation}
	\label{}
	\vec{F}=\alpha \frac{\hat{r}}{r}
\end{equation}


\subsection{Note on grad function in spherical coordinates}
\begin{equation}
	\label{}
	\nabla f =\hat{r}\frac{\partial{f}}{\partial{r}}+\hat{\theta}\frac{1}{r}\frac{\partial{f}}{\partial{\theta}}+\hat{\phi}\frac{1}{r\sin{\theta}}\frac{\partial{f}}{\partial{\phi}}
\end{equation}

\subsection{Total Work}
\begin{equation}
	\label{}
	W_{tot}=-dU
\end{equation}
\begin{equation}
	\label{}
d(T+U=0),\quad E=T+U=T_{1}+T_{2}+U
\end{equation}

\textbf{question for class } hat does it mean to take the gradient with respect to each particles coordinates?

\begin{equation}
	\label{}
	-\nabla_{\alpha}U=F_{\mbox{net on }\alpha}	
\end{equation}

\section{Simple Harmonic Motion}

If $\frac{\omega_{x}}{\omega_{y}}$ is a rational number, fraction of integers, the orbit is \textbf{closed}.
\begin{equation}
	\label{}
	F=m\ddot{x}
\end{equation}
\begin{equation}
	\label{}
	m\ddot{x}+2\beta\dot{x}+\omega_{0}^{2}+\frac{F(t)}{m},\quad \frac{F(t)}{m}=f(m)
\end{equation}
\begin{equation}
	\label{}
	\beta=\frac{b}{2m},\quad \omega_{0}=\sqrt{\frac{k}{m}} 
\end{equation}
If $f(t)=0$
\begin{equation}
	\label{}
	x(t)=C_{1}e^{r_{1}t}+C_{2}e^{r_{2}t},\quad r_{1},r_{2}\in\mathbb{R}
\end{equation}
\begin{equation}
	\label{}
	r_{1}=-\beta+{\sqrt{\beta^{2}-\omega_{0}^{2}}}
\end{equation}
\begin{equation}
	\label{}	
	r_{2}=-\beta-{\sqrt{\beta^{2}-\omega_{0}^{2}}}
\end{equation}


\subsection{Damping}


for $\beta>\omega_{0}$ system is \textbf{over damped}, for $w_{0}>\beta$ oscillator is \textbf{under damped}, and if $\omega_{0}=\beta$ the oscillator is \textbf{critically damped.}

\hfill
\hfill

Under Damped
\begin{equation}
	\label{}
	x(t)=C_{1}e^{r_{1}t}+C_{2}e^{r_{2}t},\quad r_{1},r_{2}\in\mathbb{R}
\end{equation}
Under Damped
\begin{equation}
	\label{}
x(t)=Ae^{-\beta t}\cos(\omega_{1}t+\delta)
\end{equation}
\begin{equation}
	\label{}
	\omega_{1}=\sqrt{\omega_{0}^{2}-\beta^{2}}
\end{equation}
Critically Damped
\begin{equation}
	\label{}
	x(t)=C_{1}e^{-\beta t}+C_{2}te^{-\beta t}
\end{equation}


\subsection{Driven Under damped Oscillations}
define
\begin{equation}
	\label{}
D=\frac{d^{2}}{dt^{^2}}+2\beta \frac{d}{dt}+\omega_{0}^{2}
\end{equation}


\section{Hamiltonian Mechanics}
\begin{equation}
	\label{}
	S=\int_{t_{1}}^{t}\mathcal{L}dt 
\end{equation}
This is \textbf{Hamilton's Principal.}
With S being action and $\mathcal{L}$ being the Lagrangian. Motion will follow the path that \textbf{minimizes} this integral.  

\begin{equation}
	\label{}
	\mathcal{L}=T-U=\frac{1}{2}m\dot{x}^{2}-U(x)
\end{equation}
\begin{equation}
	\label{}
	\mathcal{L}=\mathcal{L}(x,\dot{x},t)
\end{equation}
\begin{equation}
	\label{}
	S(x)=\int_{t_{1}}^{t_{2}}\mathcal{L}(x+\alpha\eta,\dot{x}\alpha\dot{\eta},t)
\end{equation}
\begin{equation}
	\label{}
	\delta S = \frac{dS(\alpha)}{d\alpha}\bigg|_{\alpha\to 0}=\lim\int_{t_{1}}^{t_{2}}\bigg(\frac{\partial\mathcal{L}}{\partial(x)}\eta +\frac{\partial \mathcal{L}}{\partial(\dot{x})}\dot{\eta}\bigg)dt
\end{equation}
Integrated by parts
\subsection{Euler Lagrange Equation}
This one is usually used.
\begin{equation}
	\label{//}
	\frac{\partial\mathcal{L}}{\partial x}-\frac{d}{dt}\frac{\partial\mathcal{L}}{\partial{\dot{x}}}=0
\end{equation}

\subsection{Line Elements}
\begin{equation}
	\label{}
	ds=\sqrt{d\hat{r}\cdot d\hat{r}}
\end{equation}

\textbf{Spherical Polar}
\begin{equation}
	\label{}
	d\vec{r}=dr\hat{r}+rd\theta\hat{\theta}+r\sin\theta d\phi\hat{\phi}	
\end{equation}

\textbf{Cylindrical}
\begin{equation}
	\label{}
	d\hat{r}=d\rho\hat{\rho}+\rho d\phi\hat{\phi}+dz\hat{z}
\end{equation}

\textbf{Polar}
\begin{equation}
	\label{}
	d\hat{r}=dr\hat{r}+rd\theta\hat{\theta}
\end{equation}
\subsection{Cyclic and ignorable general coordinates}
for general coordinates:
\begin{equation}
	\label{}
	\frac{\partial\mathcal{L}}{q_{i}}=\frac{d}{dt}\frac{\partial\mathcal{L}}{\partial\dot{q_{i}}}
\end{equation}
when $\frac{\partial\mathcal{L}}{\partial q_{i}}=0$, $q_{i}$ is a \textbf{cyclic} or \textbf{ignorable general coordinate.} This $\frac{\partial\mathcal{L}}{\partial \dot{q_{i}}}$ is then a constant, let integration and derivation follow.

\subsection{Finding the Lagrangian/Conservation}
\textbf{Translational Invariance: } $x\to x+\epsilon$, $\mathcal{L}(x+\epsilon,\dot{x},t)=\mathcal{L}(x,\dot{x},t)$, where $\epsilon$ is a small transformation. That momentum is conserved follows.
\begin{equation}
	\label{}
	\delta\mathcal{L}\equiv\mathcal{L}(x+\epsilon,\dot{x},t)-\mathcal{L}(x,\dot{x},t)=0
\end{equation}

\textbf{Time Translation: } $t\to t+\delta t$.
\begin{equation}
	\label{}
	\delta\mathcal{L}=0=\frac{\partial\mathcal{L}}{\partial t}\delta t,\quad \frac{\partial\mathcal{L}}{\partial t}=0
\end{equation}
\begin{equation}
	\label{}
	\frac{d\mathcal{L}}{dt}=\frac{\partial\mathcal{L}}{\partial x}\frac{dx}{dt}+\frac{\partial\mathcal{L}}{\partial\dot{x}}\frac{d\dot{x}}{dt}+\frac{\partial\mathcal{L}}{\partial t}
\end{equation}
reduces to
\begin{equation}
	\label{}
	\frac{\partial\mathcal{L}}{\partial t}=\frac{d}{dt}(p\dot{x}-\mathcal{L})
\end{equation}

\begin{equation}
	\label{}
	\mathcal{H}=\frac{\partial\mathcal{L}}{\partial \dot{q_{i}}}\dot{x}-\mathcal{L}
\end{equation}
In simple cases the Hamiltonian, $\mathcal{H}$, is the total energy of the system, but not always. It's often a first order differential equation, which is relatively easy to solve. If the Hamiltonian is conserved it gives a convenient way of solving the problem. 

\begin{enumerate}
	\item Write down $T and U\to\mathcal{L}=T-U$
	\item Pick out generalized coordinates consistent with constraints
	\item Write Lagrangian in generalized coordinates
	\item Identify any cyclic coordinates
		\begin{enumerate}
			\item If $\frac{\partial\mathcal{L}}{\partial t}=0$, find $\mathcal{H}	$
		\end{enumerate}
	\item Use Euler-Lagrange
\end{enumerate}

\section{Two Body Central Force Problems}

Two masses, $m_{1},m_{2}$. Only force is the force between them. Only force is the conservative and central force from and between both bodies. 

For forces like gravity and the Coulomb force the potential is a function of the magnitude of the vector between the two bodies, $|r_{1}-r_{2}|$. For potentials such as these, they are \textbf{translational invariants}. It is convenient to represent $|r_{1}-r_{2}$ as $\vec{r}$, the relative position.

\hfill
\hfill

\textbf{r}  is then used as one of our coordinates. Center of Mass is the other we will use,
\begin{equation}
	\label{}
	R=\frac{m_{1}r_{1}+m_{2}r_{2}}{m_{1}+m_{2}},\quad M=m_{1}+m_{2}
\end{equation}
The momentum of the system
\begin{equation}
	\label{}
	P=M\dot{R}=0
\end{equation}
Is conserved, and therefore constant, and we can chose a reference frame in which it as at rest and equals zero. The \textbf{reduced mass} is a useful parameter.
\begin{equation}
	\label{}
	\mu=\frac{m_{1}m_{2}}{M}
\end{equation}
this allows us to write
\begin{equation}
	\label{}
	T=\frac{1}{2}M\dot{R}^{2}+\frac{1}{2}\mu\dot{r}^{2}
\end{equation}
This shows that the kinetic energy is equivalent to a fictitious system with two particles, one  of mass M moving at the speed of the center of mass, and the other of mass $\mu$ moving with speed of the relative position \textbf{r}. 
\begin{equation}
	\label{}
	\mathcal{L}=T-U=\frac{1}{2}M\dot{R}^{2}+\bigg(\frac{1}{2}\mu\dot{r}^{2}-U(r)\bigg)=\mathcal{L}_{cm}+\mathcal{L}_{rel}
\end{equation}
Separating the Lagrangian allows the solution of two independent functions, one a function of R, and one a function of r. This is convenient because we can set our reference frame according to R. 

Position of either particle:
\begin{equation}
	\label{}
	r_{i}=\frac{m_{j}}{M}r
\end{equation}

Angular Momentum
\begin{equation}
	\label{}
	L=r \times \mu\dot{r}
\end{equation}
Because we know angular momentum is conserved, the direction of $r \times \dot{r}$ is constant, and thus all motion happens in a plane. 

\textbf{From the Center of Mass point of view} the two orbiting bodies have equal momenta, so their masses gives relative velocities. 

\begin{equation}
	\label{}
	\mu r \dot{\phi}^{2}-\frac{dU}{dr}=\mu\ddot{r}
\end{equation}
Which can easily be rearranged for force.
\begin{equation}
	\label{}
	\dot{\phi}=\frac{l}{\mu r^{2}}
\end{equation}
Useful to substitute into preceding. 
\begin{equation}
	\label{}
	U_{eff}(r)=U(r)+U_{cf}(r)=U(r)+\frac{l^{2}}{2\mu r^{2}}
\end{equation}

\begin{equation}
	\label{}
	r(\phi)=\frac{c}{1+\epsilon\cos{\phi}}
\end{equation}
where $c=\frac{l^{2}}{\gamma mu}$


\hfill
Relationship between eccentricity, $\epsilon$, and total energy. Negative energies give $\epsilon<1$ and bounded orbits. Positive energies give $\epsilon>1$ and unbounded orbits 1
\begin{equation}
	\label{}
	E=\frac{\gamma^{2}\mu}{2l^{2}}(\epsilon^{2}-1)
\end{equation}



\section{Tensors}
\subsection{Some vector identities}

\begin{equation}
	\label{}
	(a \times  b)\cdot (c \times d)=(a\cdot c)(b\cdot d)-(b\cdot c)(a\cdot d)
\end{equation}
\begin{equation}
	\label{}
	a\cdot(b \times c)=b\cdot(c \times a)=c\dot(a \times b)
\end{equation}
\begin{equation}
	\label{}
	a \times (b \times c)=(c\cdot a)b=c(a\cdot b)
\end{equation}

\begin{equation}
	\label{}
	\frac{dQ}{dt}_{s_{0}}=\frac{dQ}{dt}_{s}+\Omega \times Q
\end{equation}
\begin{equation}
	\label{}
	m\ddot{\vec{r}}=-\nabla U - m \Omega \times (\Omega \times r)-2m(\Omega \times v)
\end{equation}
Where the second term after U is the \textbf{centrifugal force} and the last term is the \textbf{Coriolis force}

Omega points in the \textbf{axis of rotation} and the magnitude represents the angular velocity.

\section{Non-Inertial Reference Frames}
In a reference frame with acceleration:
\begin{equation}
	\label{}
	m\ddot{r}=F-mA,\quad F_{\mbox{inertial}}=-mA
\end{equation}

\begin{equation}
	\label{}
	\vec{\omega}=\omega\vec{u}
\end{equation}
Where u is the axis of rotation and $\omega$ is the angular velocity. 


\hfill

The velocity of any point or vector rotating on an axis is given by
\begin{equation}
	\label{}
	v=\vec{\omega} \times r,\quad \frac{de}{dt}=\omega \times e
\end{equation}



Useful Identity
\begin{equation}
	\label{}
	m(\Omega \times r) \times  \Omega = m\Omega^{2}\rho\hat{\rho}
\end{equation}

\section{Rotational Momentum}
For a \textbf{symmetric} rotating system $L=I\omega$, $I=\sum(m_{i}r_{\perp i}^{2})$

\begin{equation}
	\label{}
	I=\int r_{\perp}^{2} dm
\end{equation}



\end{document}

