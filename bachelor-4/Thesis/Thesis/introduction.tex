%% Photoionization 
%% BPT - list the ion ratios, include square bracket and why its 'forbidden' 



Attempts of understanding the nature, structure, and evolution of galaxies have been made since the inception of the concept of the galaxy in the early 20th century. Early taxonomies based on morphology have gradually been replace with other categorization schemes intended to more accurately reflect the character of the galaxy. Currently galaxies are typically categorized based on the dominant radiation we observe from them into one of three groups: Active Galactic Nuclei (AGN), regions of active star formation, and LI(N)ERs.\cite{bpt-paper} Of these three galaxy types, the ionizing radiation source for LI(N)ERs is by far the least well defined.\cite{byler-emission-line-predictions} LI(N)ER is a descriptive name, Low Ionization (Nuclear) Emission Region, and corresponds to large ionized clouds with relatively low levels of ionization whose emission and absorption lines we can measure from Earth.  Previous investigations have identified a broad class of stars known as 'post-AGN stars' that may be responsible for producing emission spectra characteristic of LI(N)ER-type galaxies\cite{byler-emission-line-predictions}.


  Previous investigation of LI(N)ERs has been limited by the distance light must travel in the massive scale of the extragalactic universe, which has restricted the ability to make detailed predictions regarding these systems. Luckily recent optical and ultraviolet observations of the inner Milky Way has found LI(N)ER-like emission spectra.\cite{inner-galaxy-liner-like-gas} This opens a huge new opportunity to study LI(N)ER-type galaxies in much greater detail and further constrain the radiation source and stellar population of these galaxies.


In this project we are working to constrain the inner radiation field of the Milky Way by identifying star types that could be responsible for the ionization we see. To do this we model a specific star type irradiating a celestial cloud of gas, causing it to ionize, and then compare the resultant emission and absorption lines to what we have measured in the Milky Way and other galaxies.


