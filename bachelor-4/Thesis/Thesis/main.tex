\documentclass[twocolumn]{sintr}
\setlength\parindent{0pt}
\usepackage{lipsum}  

% 
% Volume, issue, year, page range
\articleinformation{2023} 
% 
%
% BSMS Thesis / Winter Project / Summer Project / Project
\articletype{Colorado College Physics Senior Thesis} 
%
\submitted{20 January 2023} 
%
% Article title
\title{Do sub-dwarf OB or white-dwarf stars of $ ~10^{4} $ Kelvin produce LI(N)ER-like outputs in ionized gas?}
%
\articleshorttile{What stars are in LI(N)ERs?}
% Author(s)
%
\author[1]{Owen Cox}
%
% Author name(s) as they should appear in the header
\authorheader{Owen Cox}
%
%
%% ==>
%
%--------------------------------------------------
% Author(s) affiliations                      -->  
%--------------------------------------------------
%
%% ==>

%
\affil[1]{1F5 - AB2, Department of Physics, Owen Cox}
%% ==>

% Corresponding author
\affil[$\mbox{*}$]{Corresponding author: \href{mailto:o_cox@coloradocollege.edu}{o\_cox@coloradocollege.edu}}
%
% Abstract
\abstract{Galaxies are typically classified based on the dominant form of radiation they emit into one of three main groups: active galactic nuclei (AGN), regions of active star formation, and low-ionization (nuclear) emission regions (LI(N)ERs). Our understanding of the processes that cause LI(N)ER-like emissions is the most lacking. The radiation source of this ionized gas and by extension the population of stars of these regions is poorly constrained. However, recent optical observations of the Tilted Disk in the Milky Way’s center from the Wisconsin H-Alpha Mapper (WHAM) telescope reveal emission line ratios in the central region of the Milky Way similar to that of other LI(N)ER galaxies, giving the opportunity for a much more detailed study of LI(N)ERs. Here we will identify the conditions that create emission line ratios similar to those observed, focusing at first on individual stellar sources, such as hot white dwarfs, or sub-dwarf OB (sdOB) stars . Our initial parameter study will provide a fundamental framework to expand into more complex stellar populations and develop a full understanding of the ionizing radiation responsible for LI(N)ERs. This structural understanding is essential in our search to decipher the formation and evolution of all galaxies, and could shed light even on the relationships between AGN, galaxies with active star formation, and LI(N)ERs.}

% Keywords
\keywords{Alphabetical order \\ 
Maximum five keywords \\
Avoid terms already in \\
\hphantom{} the title}

% PDF metadata (authors ignore)
\makeatletter
\hypersetup{pdftitle={\@title},pdfauthor={\authorheader}, pdfkeywords={\keywords}, pdfsubject={\abstract}}
\makeatother

\makeatletter
\def\MT@warn@unknown{}
\makeatother



\begin{document}
\setcounter{page}{1}
\maketitle

\thispagestyle{firststyle}
%\linenumbers % Add line numbers

\section{Introduction}

%% Photoionization 
%% BPT - list the ion ratios, include square bracket and why its 'forbidden' 



Attempts of understanding the nature, structure, and evolution of galaxies have been made since the inception of the concept of the galaxy in the early 20th century. Early taxonomies based on morphology have gradually been replace with other categorization schemes intended to more accurately reflect the character of the galaxy. Currently galaxies are typically categorized based on the dominant radiation we observe from them into one of three groups: Active Galactic Nuclei (AGN), regions of active star formation, and LI(N)ERs.\cite{bpt-paper} Of these three galaxy types, the ionizing radiation source for LI(N)ERs is by far the least well defined.\cite{byler-emission-line-predictions} LI(N)ER is a descriptive name, Low Ionization (Nuclear) Emission Region, and corresponds to large ionized clouds with relatively low levels of ionization whose emission and absorption lines we can measure from Earth.  Previous investigations have identified a broad class of stars known as 'post-AGN stars' that may be responsible for producing emission spectra characteristic of LI(N)ER-type galaxies\cite{byler-emission-line-predictions}.


  Previous investigation of LI(N)ERs has been limited by the distance light must travel in the massive scale of the extragalactic universe, which has restricted the ability to make detailed predictions regarding these systems. Luckily recent optical and ultraviolet observations of the inner Milky Way has found LI(N)ER-like emission spectra.\cite{inner-galaxy-liner-like-gas} This opens a huge new opportunity to study LI(N)ER-type galaxies in much greater detail and further constrain the radiation source and stellar population of these galaxies.


In this project we are working to constrain the inner radiation field of the Milky Way by identifying star types that could be responsible for the ionization we see. To do this we model a specific star type irradiating a celestial cloud of gas, causing it to ionize, and then compare the resultant emission and absorption lines to what we have measured in the Milky Way and other galaxies.




\section{Methods}

To determine what stars could be responsible for these LI(N)ER-type emissions we use the popular photoionization model Cloudy. 


\section{Various Commands}
\subsection{Tables}

Size a table to fit in a single column (Table \ref{tab:1}) or across two columns (Table \ref{tab:2}). Avoid large tables (i.e., those that fit more than a single page), unless necessary.

Every table and figure should be cited in the text in numerical order (i.e., Table 2 cannot be cited before Table 1). Place table footnotes below the table, indicating them with superscripted lowercase letters or asterisks (for significance values and other statistical data).

%
%
%---------------------------------------------
%

\subsubsection{Table captions}
Every table should have a concise but clear caption to explain its main components independently from the text. If the table contains previously published material, cite the original source at the end of the caption. If the results are expressed as a percentage, state the absolute value(s) that correspond to 100\%.

\begin{table}[b]
  	\centering\footnotesize\sffamily
  	\caption{Example single-column table.}
  	\begin{tableminipage}{\linewidth}
    	\begin{tabularx} {\linewidth}{XXX}
			\toprule
            Column 1\textsuperscript{a} & Column 2 & Column 3  \\            
	    	\midrule
             1 &  1 &  1 \\
             2 &  2 &  2 \\
             3 &  3 &  3 \\
             4 &  4 &  4 \\
             5 &  5 &  5 \\
            \bottomrule
    	\end{tabularx}
        \label{tab:1}
        \vskip0pt
        \textsuperscript{a}Example footnote.
  	\end{tableminipage}
\end{table}

\begin{table*}[t]
  	\centering\footnotesize\sffamily
  	\caption{Example double-column table.}
  	\begin{tableminipage}{\linewidth}
    	\begin{tabularx} {\linewidth}{XXXXXX}
			\toprule
            Column 1 & Column 2 & Column 3 & Column 4 & Column 5 & Column 6 \\            
	    	\midrule
             11 &  12 &  13 &  14 &  15 &  16 \\
             21 &  22 &  23 &  24 &  25 &  26 \\
             31 &  32 &  33 &  34 &  35 &  36 \\
             41 &  42 &  43 &  44 &  45 &  46 \\
             51 &  52 &  53 &  54 &  55 &  56 \\
            \bottomrule
    	\end{tabularx}
        \label{tab:2}
  	\end{tableminipage}
\end{table*}

\subsection{Figures}

Ensure that the figure will fit into either one column (Figure \ref{fig:1}) or two columns (Figure \ref{fig:1}). Images should be of sufficiently high resolution to be easily viewable when printed or on high-resolution screens (minimum of 300 dpi).

Every figure should be cited in the text in numerical order (i.e. Figure 2 cannot be cited before Figure 1). Figures should be referred to as "Figure" not "Fig." Denote figure parts with lowercase letters (e.g. Figure 1a, Figure 1b).

\subsubsection{Figure formatting}

Photographs must have internal scale markers and symbols; arrows or letters should contrast greatly with the background. Fira Sans is the recommended typeface for text within figures (if you don’t have it installed on your computer, you can download it from Google Fonts). Otherwise, a sans-serif such as Open Sans or Helvetica may be used. Where photographs of gel, autoradiograms, and so on have been processed to enhance their quality, this should be stated.


\lipsum[1]

\subsubsection{Figure captions}

Every figure should have a concise but clear caption to explain its main components independently from the text. If the figure contains previously published material, cite the source at the end of the caption.

\begin{figure}[!b]
	\centering
	\includegraphics[width=\linewidth]{Figure1.pdf}
	\caption{Example figure caption for a single-column image.}
	\label{fig:1}
\end{figure}

\lipsum[1]


\section{Conclusion}

Present the study's main conclusions, along with their implications for future research or science and technology policy in the ASEAN region.

\section*{Acknowledgement}

Please acknowledge anyone who contributed to the research or the writing of the manuscript, as well as any funding or grants received in support of it. The names of funding organizations should be written in full, along with the grant numbers, if available. Examples of individuals you should acknowledge include people who assisted with study design or analysis, or guidance through a study area, or who provided advice on the language, edited, or proofread the article.

\lipsum[2-10]

\section*{AUTHORS’ CONTRIBUTIONS}

Each author’s contribution to the research and manuscript should be noted, using only their initials to indicate their names. For example, “MP, FW designed the study. MP, LS carried out the laboratory work. MP, FW, LS, DN analyzed the data. MP, FW, DN wrote the manuscript. All authors read and approved the final version of the manuscript.”

\section*{COMPETING INTERESTS}

All competing interests—financial, professional, or personal relationships relevant to the submitted work—must be declared. This should be clearly stated if a funding source contributed to the manuscript's design, data collection, analysis, or writing or the decision to submit it to AJSTD. If one or more authors have any form of—past or present—relationship with AJSTD, the extent of this relationship must be described. This must also be clearly stated if one or more authors work or have worked for an organization that may benefit from the article's publication. Please read AJSTD’s Publication Ethics statement to understand why it is essential to acknowledge any competing interests.

\lipsum[1]

\lipsum[2-14]

\section*{References and Citations}
Please put the references in a separate file, like ref. Bib like this template. You must use the Bibtex format. Preparing the Bibtex reference format is very simple; you can use inspire hep.net  to get the reference in BibTex format. Then use \cite{aa} to cite the references  \cite{bb}. 

\bibliography{ref}

\atColsEnd{\vfill}

\end{document}
